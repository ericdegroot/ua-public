% based on the example by Philippe Dreuw and Thomas Deselaers
\documentclass[final]{beamer}
\mode<presentation> {
  \usetheme{UA-ECE}
}
\usepackage{times}
\usepackage{amsmath,amsthm, amssymb, latexsym}
\boldmath
\usepackage[english]{babel}
\usepackage[latin1]{inputenc}
%\usepackage[orientation=portrait,size=a0,scale=1.4,debug]{beamerposter}
\usepackage[orientation=portrait,size=a1,scale=1.4,debug]{beamerposter}

%%%%%%%%%%%%%%%%%%%%%%%%%%%%%%%%%%%%%%%%%%%%%%%%%%%%%%%%%%%%%%%%%%%%%%%%%%%%%%%%%5
%\graphicspath{{figures/}}
\title[Fancy Posters]{Making Really Fancy Posters with \LaTeX}
\author[Dreuw \& Deselaers]{Philippe Dreuw and Thomas Deselaers}
\institute[RWTH Aachen University]{Human Language Technology and Pattern Recognition, RWTH Aachen University}
\date{Jul. 31th, 2007}

%%%%%%%%%%%%%%%%%%%%%%%%%%%%%%%%%%%%%%%%%%%%%%%%%%%%%%%%%%%%%%%%%%%%%%%%%%%%%%%%%5
\begin{document}
\begin{frame}{} 
  \vfill
  \begin{block}{Fontsizes}
    \centering
    {\tiny tiny}\par
    {\scriptsize scriptsize}\par
    {\footnotesize footnotesize}\par
    {\normalsize normalsize}\par
    {\large large}\par
    {\Large Large}\par
    {\LARGE LARGE}\par
    {\veryHuge VeryHuge}\par
    {\VeryHuge VeryHuge}\par
    {\VERYHuge VERYHuge}\par
  \end{block}
  \vfill
  \begin{columns}[t]
    \begin{column}{.48\linewidth}
      \begin{block}{Introduction}
        \begin{itemize}
        \item some items
        \item some items
        \item some items
        \item some items
        \end{itemize}
      \end{block}
    \end{column}
    \begin{column}{.48\linewidth}
      \begin{block}{Introduction}
        \begin{itemize}
        \item some items and $\alpha=\gamma, \sum_{i}$
        \item some items
        \item some items
        \item some items
        \end{itemize}
        $$\alpha=\gamma, \sum_{i}$$
      \end{block}
      \begin{block}{Introduction}
        \begin{itemize}
        \item some items
        \item some items
        \item some items
        \item some items
        \end{itemize}
      \end{block}

      \begin{block}{Introduction}
        \begin{itemize}
        \item some items and $\alpha=\gamma, \sum_{i}$
        \item some items
        \item some items
        \item some items
        \end{itemize}
        $$\alpha=\gamma, \sum_{i}$$
      \end{block}
    \end{column}
  \end{columns}
\end{frame}
\end{document}
